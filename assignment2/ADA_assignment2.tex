% Options for packages loaded elsewhere
\PassOptionsToPackage{unicode}{hyperref}
\PassOptionsToPackage{hyphens}{url}
\PassOptionsToPackage{dvipsnames,svgnames,x11names}{xcolor}
%
\documentclass[
  letterpaper,
  DIV=11,
  numbers=noendperiod]{scrartcl}

\usepackage{amsmath,amssymb}
\usepackage{iftex}
\ifPDFTeX
  \usepackage[T1]{fontenc}
  \usepackage[utf8]{inputenc}
  \usepackage{textcomp} % provide euro and other symbols
\else % if luatex or xetex
  \usepackage{unicode-math}
  \defaultfontfeatures{Scale=MatchLowercase}
  \defaultfontfeatures[\rmfamily]{Ligatures=TeX,Scale=1}
\fi
\usepackage{lmodern}
\ifPDFTeX\else  
    % xetex/luatex font selection
\fi
% Use upquote if available, for straight quotes in verbatim environments
\IfFileExists{upquote.sty}{\usepackage{upquote}}{}
\IfFileExists{microtype.sty}{% use microtype if available
  \usepackage[]{microtype}
  \UseMicrotypeSet[protrusion]{basicmath} % disable protrusion for tt fonts
}{}
\makeatletter
\@ifundefined{KOMAClassName}{% if non-KOMA class
  \IfFileExists{parskip.sty}{%
    \usepackage{parskip}
  }{% else
    \setlength{\parindent}{0pt}
    \setlength{\parskip}{6pt plus 2pt minus 1pt}}
}{% if KOMA class
  \KOMAoptions{parskip=half}}
\makeatother
\usepackage{xcolor}
\setlength{\emergencystretch}{3em} % prevent overfull lines
\setcounter{secnumdepth}{-\maxdimen} % remove section numbering
% Make \paragraph and \subparagraph free-standing
\makeatletter
\ifx\paragraph\undefined\else
  \let\oldparagraph\paragraph
  \renewcommand{\paragraph}{
    \@ifstar
      \xxxParagraphStar
      \xxxParagraphNoStar
  }
  \newcommand{\xxxParagraphStar}[1]{\oldparagraph*{#1}\mbox{}}
  \newcommand{\xxxParagraphNoStar}[1]{\oldparagraph{#1}\mbox{}}
\fi
\ifx\subparagraph\undefined\else
  \let\oldsubparagraph\subparagraph
  \renewcommand{\subparagraph}{
    \@ifstar
      \xxxSubParagraphStar
      \xxxSubParagraphNoStar
  }
  \newcommand{\xxxSubParagraphStar}[1]{\oldsubparagraph*{#1}\mbox{}}
  \newcommand{\xxxSubParagraphNoStar}[1]{\oldsubparagraph{#1}\mbox{}}
\fi
\makeatother

\usepackage{color}
\usepackage{fancyvrb}
\newcommand{\VerbBar}{|}
\newcommand{\VERB}{\Verb[commandchars=\\\{\}]}
\DefineVerbatimEnvironment{Highlighting}{Verbatim}{commandchars=\\\{\}}
% Add ',fontsize=\small' for more characters per line
\usepackage{framed}
\definecolor{shadecolor}{RGB}{241,243,245}
\newenvironment{Shaded}{\begin{snugshade}}{\end{snugshade}}
\newcommand{\AlertTok}[1]{\textcolor[rgb]{0.68,0.00,0.00}{#1}}
\newcommand{\AnnotationTok}[1]{\textcolor[rgb]{0.37,0.37,0.37}{#1}}
\newcommand{\AttributeTok}[1]{\textcolor[rgb]{0.40,0.45,0.13}{#1}}
\newcommand{\BaseNTok}[1]{\textcolor[rgb]{0.68,0.00,0.00}{#1}}
\newcommand{\BuiltInTok}[1]{\textcolor[rgb]{0.00,0.23,0.31}{#1}}
\newcommand{\CharTok}[1]{\textcolor[rgb]{0.13,0.47,0.30}{#1}}
\newcommand{\CommentTok}[1]{\textcolor[rgb]{0.37,0.37,0.37}{#1}}
\newcommand{\CommentVarTok}[1]{\textcolor[rgb]{0.37,0.37,0.37}{\textit{#1}}}
\newcommand{\ConstantTok}[1]{\textcolor[rgb]{0.56,0.35,0.01}{#1}}
\newcommand{\ControlFlowTok}[1]{\textcolor[rgb]{0.00,0.23,0.31}{\textbf{#1}}}
\newcommand{\DataTypeTok}[1]{\textcolor[rgb]{0.68,0.00,0.00}{#1}}
\newcommand{\DecValTok}[1]{\textcolor[rgb]{0.68,0.00,0.00}{#1}}
\newcommand{\DocumentationTok}[1]{\textcolor[rgb]{0.37,0.37,0.37}{\textit{#1}}}
\newcommand{\ErrorTok}[1]{\textcolor[rgb]{0.68,0.00,0.00}{#1}}
\newcommand{\ExtensionTok}[1]{\textcolor[rgb]{0.00,0.23,0.31}{#1}}
\newcommand{\FloatTok}[1]{\textcolor[rgb]{0.68,0.00,0.00}{#1}}
\newcommand{\FunctionTok}[1]{\textcolor[rgb]{0.28,0.35,0.67}{#1}}
\newcommand{\ImportTok}[1]{\textcolor[rgb]{0.00,0.46,0.62}{#1}}
\newcommand{\InformationTok}[1]{\textcolor[rgb]{0.37,0.37,0.37}{#1}}
\newcommand{\KeywordTok}[1]{\textcolor[rgb]{0.00,0.23,0.31}{\textbf{#1}}}
\newcommand{\NormalTok}[1]{\textcolor[rgb]{0.00,0.23,0.31}{#1}}
\newcommand{\OperatorTok}[1]{\textcolor[rgb]{0.37,0.37,0.37}{#1}}
\newcommand{\OtherTok}[1]{\textcolor[rgb]{0.00,0.23,0.31}{#1}}
\newcommand{\PreprocessorTok}[1]{\textcolor[rgb]{0.68,0.00,0.00}{#1}}
\newcommand{\RegionMarkerTok}[1]{\textcolor[rgb]{0.00,0.23,0.31}{#1}}
\newcommand{\SpecialCharTok}[1]{\textcolor[rgb]{0.37,0.37,0.37}{#1}}
\newcommand{\SpecialStringTok}[1]{\textcolor[rgb]{0.13,0.47,0.30}{#1}}
\newcommand{\StringTok}[1]{\textcolor[rgb]{0.13,0.47,0.30}{#1}}
\newcommand{\VariableTok}[1]{\textcolor[rgb]{0.07,0.07,0.07}{#1}}
\newcommand{\VerbatimStringTok}[1]{\textcolor[rgb]{0.13,0.47,0.30}{#1}}
\newcommand{\WarningTok}[1]{\textcolor[rgb]{0.37,0.37,0.37}{\textit{#1}}}

\providecommand{\tightlist}{%
  \setlength{\itemsep}{0pt}\setlength{\parskip}{0pt}}\usepackage{longtable,booktabs,array}
\usepackage{calc} % for calculating minipage widths
% Correct order of tables after \paragraph or \subparagraph
\usepackage{etoolbox}
\makeatletter
\patchcmd\longtable{\par}{\if@noskipsec\mbox{}\fi\par}{}{}
\makeatother
% Allow footnotes in longtable head/foot
\IfFileExists{footnotehyper.sty}{\usepackage{footnotehyper}}{\usepackage{footnote}}
\makesavenoteenv{longtable}
\usepackage{graphicx}
\makeatletter
\newsavebox\pandoc@box
\newcommand*\pandocbounded[1]{% scales image to fit in text height/width
  \sbox\pandoc@box{#1}%
  \Gscale@div\@tempa{\textheight}{\dimexpr\ht\pandoc@box+\dp\pandoc@box\relax}%
  \Gscale@div\@tempb{\linewidth}{\wd\pandoc@box}%
  \ifdim\@tempb\p@<\@tempa\p@\let\@tempa\@tempb\fi% select the smaller of both
  \ifdim\@tempa\p@<\p@\scalebox{\@tempa}{\usebox\pandoc@box}%
  \else\usebox{\pandoc@box}%
  \fi%
}
% Set default figure placement to htbp
\def\fps@figure{htbp}
\makeatother

\KOMAoption{captions}{tableheading}
\makeatletter
\@ifpackageloaded{caption}{}{\usepackage{caption}}
\AtBeginDocument{%
\ifdefined\contentsname
  \renewcommand*\contentsname{Table of contents}
\else
  \newcommand\contentsname{Table of contents}
\fi
\ifdefined\listfigurename
  \renewcommand*\listfigurename{List of Figures}
\else
  \newcommand\listfigurename{List of Figures}
\fi
\ifdefined\listtablename
  \renewcommand*\listtablename{List of Tables}
\else
  \newcommand\listtablename{List of Tables}
\fi
\ifdefined\figurename
  \renewcommand*\figurename{Figure}
\else
  \newcommand\figurename{Figure}
\fi
\ifdefined\tablename
  \renewcommand*\tablename{Table}
\else
  \newcommand\tablename{Table}
\fi
}
\@ifpackageloaded{float}{}{\usepackage{float}}
\floatstyle{ruled}
\@ifundefined{c@chapter}{\newfloat{codelisting}{h}{lop}}{\newfloat{codelisting}{h}{lop}[chapter]}
\floatname{codelisting}{Listing}
\newcommand*\listoflistings{\listof{codelisting}{List of Listings}}
\makeatother
\makeatletter
\makeatother
\makeatletter
\@ifpackageloaded{caption}{}{\usepackage{caption}}
\@ifpackageloaded{subcaption}{}{\usepackage{subcaption}}
\makeatother

\usepackage{bookmark}

\IfFileExists{xurl.sty}{\usepackage{xurl}}{} % add URL line breaks if available
\urlstyle{same} % disable monospaced font for URLs
\hypersetup{
  pdftitle={ADA\_assignment2},
  colorlinks=true,
  linkcolor={blue},
  filecolor={Maroon},
  citecolor={Blue},
  urlcolor={Blue},
  pdfcreator={LaTeX via pandoc}}


\title{ADA\_assignment2}
\author{}
\date{}

\begin{document}
\maketitle


\section{pdf 변환을 위한 환경 변수
수정}\label{pdf-uxbcc0uxd658uxc744-uxc704uxd55c-uxd658uxacbd-uxbcc0uxc218-uxc218uxc815}

\begin{Shaded}
\begin{Highlighting}[]
\FunctionTok{Sys.getenv}\NormalTok{(}\StringTok{"TEMP"}\NormalTok{)}
\end{Highlighting}
\end{Shaded}

\begin{verbatim}
[1] "C:\\Temp"
\end{verbatim}

\begin{Shaded}
\begin{Highlighting}[]
\FunctionTok{Sys.getenv}\NormalTok{(}\StringTok{"TMP"}\NormalTok{)}
\end{Highlighting}
\end{Shaded}

\begin{verbatim}
[1] "C:\\Temp"
\end{verbatim}

\begin{itemize}
\tightlist
\item
  경로에 한글명이 들어가면 오류
\item
  TEMP/TMP 경로를 강제로 C:/Temp 로 고정
\item
  Windows 검색 → 환경 변수 편집 -\textgreater{} 시스템 속성
  -\textgreater{} 고급 -\textgreater{} 환경변수(N)
\item
  사용자 환경 변/시스템 환경 변수에서 TEMP/TMP 모두 C:\Temp로 변경함
\end{itemize}

\section{Problem 1}\label{problem-1}

\subsection{(a)}\label{a}

\subsection{(b)}\label{b}

\subsection{(c)}\label{c}

\subsection{(d)}\label{d}

\subsection{(e)}\label{e}

\section{Problem 2}\label{problem-2}

\subsection{(a)}\label{a-1}

\begin{Shaded}
\begin{Highlighting}[]
\NormalTok{calculate\_kernel }\OtherTok{\textless{}{-}} \ControlFlowTok{function}\NormalTok{(x, rho)}
\NormalTok{\{}
\NormalTok{  n }\OtherTok{\textless{}{-}} \FunctionTok{nrow}\NormalTok{(x)}
\NormalTok{  k }\OtherTok{\textless{}{-}} \FunctionTok{matrix}\NormalTok{(}\DecValTok{0}\NormalTok{, n, n)}
  
  \ControlFlowTok{for}\NormalTok{ (i }\ControlFlowTok{in} \DecValTok{1}\SpecialCharTok{:}\NormalTok{n)}
    \ControlFlowTok{for}\NormalTok{ (j }\ControlFlowTok{in} \DecValTok{1}\SpecialCharTok{:}\NormalTok{n)}
\NormalTok{      k[i, j] }\OtherTok{\textless{}{-}} \FunctionTok{exp}\NormalTok{(}\SpecialCharTok{{-}}\NormalTok{rho }\SpecialCharTok{*} \FunctionTok{sum}\NormalTok{((x[i, ] }\SpecialCharTok{{-}}\NormalTok{ x[j, ])}\SpecialCharTok{\^{}}\DecValTok{2}\NormalTok{))}
  
  \FunctionTok{return}\NormalTok{(k)}
\NormalTok{\}}
\end{Highlighting}
\end{Shaded}

\subsection{(b)}\label{b-1}

\begin{Shaded}
\begin{Highlighting}[]
\FunctionTok{library}\NormalTok{(Rcpp)}

\FunctionTok{cppFunction}\NormalTok{(}\StringTok{"}
\StringTok{NumericMatrix calculate\_kernel\_rcpp(NumericMatrix x, double rho) \{}
\StringTok{  int n = x.nrow();}
\StringTok{  int p = x.ncol();}
\StringTok{  }
\StringTok{  NumericMatrix k(n, n);}

\StringTok{  for (int i = 0; i \textless{} n; i++) \{}
\StringTok{    for (int j = 0; j \textless{} n; j++) \{}
\StringTok{      }
\StringTok{      double dist = 0.0;}
\StringTok{      for (int t = 0; t \textless{} p; t++) \{}
\StringTok{        double diff = x(i, t) {-} x(j, t);}
\StringTok{        dist += diff * diff;}
\StringTok{      \}}
\StringTok{      }
\StringTok{      k(i, j) = exp({-}rho * dist);}
\StringTok{    \}}
\StringTok{  \}}
\StringTok{  }
\StringTok{  return k;}
\StringTok{\}}
\StringTok{"}\NormalTok{)}
\end{Highlighting}
\end{Shaded}

\subsection{(c)}\label{c-1}

\begin{Shaded}
\begin{Highlighting}[]
\FunctionTok{library}\NormalTok{(microbenchmark)}

\FunctionTok{set.seed}\NormalTok{(}\DecValTok{1}\NormalTok{)}
\NormalTok{n }\OtherTok{\textless{}{-}} \DecValTok{1000}
\NormalTok{p }\OtherTok{\textless{}{-}} \DecValTok{10}
\NormalTok{rho }\OtherTok{\textless{}{-}} \FloatTok{0.5}
\NormalTok{x }\OtherTok{\textless{}{-}} \FunctionTok{matrix}\NormalTok{(}\FunctionTok{runif}\NormalTok{(n }\SpecialCharTok{*}\NormalTok{ p), }\AttributeTok{nrow =}\NormalTok{ n, }\AttributeTok{ncol =}\NormalTok{ p)}

\CommentTok{\# 속도 비교}
\NormalTok{time\_result }\OtherTok{\textless{}{-}} \FunctionTok{microbenchmark}\NormalTok{(}
  \AttributeTok{R =} \FunctionTok{calculate\_kernel}\NormalTok{(x, rho),}
  \AttributeTok{Rcpp =} \FunctionTok{calculate\_kernel\_rcpp}\NormalTok{(x, rho),}
  \AttributeTok{times =} \DecValTok{50}
\NormalTok{)}

\NormalTok{time\_result}
\end{Highlighting}
\end{Shaded}

\begin{verbatim}
Unit: milliseconds
 expr      min       lq      mean    median       uq      max neval
    R 632.6933 642.7847 725.67191 657.05185 676.9005 1197.161    50
 Rcpp  38.5239  38.9484  50.49805  39.68145  41.2094  124.645    50
\end{verbatim}

\subsection{(d)}\label{d-1}

\begin{Shaded}
\begin{Highlighting}[]
\NormalTok{time\_table }\OtherTok{\textless{}{-}} \FunctionTok{summary}\NormalTok{(time\_result)[, }\FunctionTok{c}\NormalTok{(}\StringTok{"expr"}\NormalTok{, }\StringTok{"mean"}\NormalTok{, }\StringTok{"median"}\NormalTok{, }\StringTok{"min"}\NormalTok{, }\StringTok{"max"}\NormalTok{)]}
\NormalTok{time\_table}
\end{Highlighting}
\end{Shaded}

\begin{verbatim}
  expr      mean    median      min      max
1    R 725.67191 657.05185 632.6933 1197.161
2 Rcpp  50.49805  39.68145  38.5239  124.645
\end{verbatim}

\section{Problem 3}\label{problem-3}

\subsection{(a)}\label{a-2}

\begin{Shaded}
\begin{Highlighting}[]
\FunctionTok{set.seed}\NormalTok{(}\DecValTok{1}\NormalTok{)}
\NormalTok{n }\OtherTok{=} \DecValTok{150}
\NormalTok{X }\OtherTok{=} \FunctionTok{matrix}\NormalTok{(}\FunctionTok{runif}\NormalTok{(n,}\SpecialCharTok{{-}}\DecValTok{1}\NormalTok{, }\DecValTok{1}\NormalTok{), }\AttributeTok{ncol =} \DecValTok{1}\NormalTok{)}
\NormalTok{ftrue }\OtherTok{=} \ControlFlowTok{function}\NormalTok{(x) }\FunctionTok{sin}\NormalTok{(}\DecValTok{2}\SpecialCharTok{*}\NormalTok{pi}\SpecialCharTok{*}\NormalTok{x) }\SpecialCharTok{+} \FloatTok{0.5}\SpecialCharTok{*}\FunctionTok{cos}\NormalTok{(}\DecValTok{8}\SpecialCharTok{*}\NormalTok{pi}\SpecialCharTok{*}\NormalTok{x)}
\NormalTok{y }\OtherTok{=} \FunctionTok{ftrue}\NormalTok{(X[,}\DecValTok{1}\NormalTok{]) }\SpecialCharTok{+} \FunctionTok{rnorm}\NormalTok{(n, }\AttributeTok{sd =} \FloatTok{0.1}\NormalTok{)}
\end{Highlighting}
\end{Shaded}

\subsection{(b)}\label{b-2}

\subsection{(c)}\label{c-2}

\section{Problem 4}\label{problem-4}

\subsection{(a)}\label{a-3}

\subsection{(b)}\label{b-3}

\subsection{(c)}\label{c-3}

\subsection{(d)}\label{d-2}




\end{document}
